%
% From http://www.tug.dk/FontCatalogue/libertine/
%
% The license of the LATEX Font Catalogue is GNU General Public License.
%

%%%%%%%%%%%%%%%%%%%%%%%%%%%%%%%%%%%%%%%%%%%%%%%%%%%%%%%%%%%%%%%%%%
%%%%%%%%%%%%%%%%%%%%%%%%%%%%%%%%%%%%%%%%%%%%%%%%%%%%%%%%%%%%%%%%%%
%Packages
\documentclass[danish,a4paper,11pt]{scrartcl}
\usepackage{libertine}
\usepackage[T1]{fontenc}
\usepackage[utf8]{inputenc}
\usepackage{babel}
\usepackage{slantsc}
\usepackage{array}
\usepackage{amsthm}
\let\proof\relax
\let\endproof\relax
\usepackage{amsmath, amsfonts, amssymb, amscd, fancyhdr, color, comment, graphicx, environ, mathtools}
\usepackage{algorithm}
\usepackage{algpseudocode}
\usepackage{tabularx}
\setkomafont{subsection}{\usefont{T1}{fvm}{m}{n}}
\setkomafont{section}{\usefont{T1}{fvs}{b}{n}\Large}
\setcounter{secnumdepth}{0}
\pagestyle{empty}
\usepackage{xspace}




%%%%%%%%%%%%%%%%%%%%%%%%%%%%%%%%%%%%%%%%%%%%%%%%%%%%%%%%%%%%%%%%%%
%%%%%%%%%%%%%%%%%%%%%%%%%%%%%%%%%%%%%%%%%%%%%%%%%%%%%%%%%%%%%%%%%%
%Add new commands here
\newcommand{\Z}{\mathbb Z}
\newcommand{\R}{\mathbb R}
\newcommand{\Q}{\mathbb Q}
\newcommand{\NN}{\mathbb N}
\newcommand{\PP}{\mathbb P}

% Add a period to the end of an abbreviation unless there's one
% already, then \xspace.
\makeatletter
\DeclareRobustCommand\onedot{\futurelet\@let@token\@onedot}
\def\@onedot{\ifx\@let@token.\else.\null\fi\xspace}

\def\eg{\emph{e.g}\onedot} \def\Eg{\emph{E.g}\onedot}
\def\ie{\emph{i.e}\onedot} \def\Ie{\emph{I.e}\onedot}
\def\cf{\emph{c.f}\onedot} \def\Cf{\emph{C.f}\onedot}
\def\etc{\emph{etc}\onedot} \def\vs{\emph{vs}\onedot}
\def\wrt{w.r.t\onedot} \def\dof{d.o.f\onedot}
\def\etal{\emph{et al}\onedot}

\newcommand{\solution}{\noindent\underline{\textbf{Solution.}} }
\newcommand{\remark}{\textbf{Remark.} }

\newcounter{thmcnt}
\newcounter{lmacnt}
\newcounter{defcnt}
\newcounter{expcnt}
\newsavebox{\Defname}
\newenvironment{theorem}{
    \stepcounter{thmcnt}
    \textsc{Theorem \arabic{thmcnt}.}
}{\hfill\par}
\newenvironment{lemma}{
    \stepcounter{lmacnt}
    \textsc{Lemma \arabic{lmacnt}.}
}{\hfill\par}
\newenvironment{definition}[1]
{
    \sbox\Defname{\emph{#1}}
    \stepcounter{defcnt}
    \textsc{Definition \arabic{defcnt}} (\usebox{\Defname}).
}
{\hfill\par}
\newenvironment{proof}{
    \textit{Proof.}
}{\hfill\par}
\newenvironment{example}{
    \stepcounter{expcnt}
    \textsc{Example \arabic{expcnt}.}
}{\hfill\par}


\begin{document}

\begin{titlepage}
    \begin{center}
        \vspace*{3cm}
            
        \Huge
        \textbf{Course Name}
            
        \vspace{1cm}
        \huge
        Assignment 
            
        


            
        \vspace{1.5cm}
        \Large
            
        \textbf{Author}                      % <-- author
        
            
        \vfill
        
       
     
        \includegraphics[width=0.8\textwidth]{pics/CSE_logo.pdf}
        \\
        
 
            
    \end{center}
\end{titlepage}


\noindent\underline{\textbf{Problem1.}} \textsf{\color{blue} Prove that $P=NP$.}

\solution In progress. 

\remark It's known as one of "The Seven Difficulties of Millennials". \\


\noindent\underline{\textbf{Problem2.}} \textsf{\color{blue} State the differential privacy.}

\solution A randomized algorithm $\mathcal{M}$ with domain $\NN^{|\chi|}$ is $(\epsilon, \delta)$-differential private if for all $\mathcal{S} \in \textnormal{Range}(\mathcal{M})$ and for all $x, y \in \NN^{|\chi|}$ such that $||x-y||_1 \leq 1$:
\begin{align} \label{equ:dp}
    \Pr[\mathcal{M}(x) \in \mathcal{S}] \leq \exp{(\epsilon)} \Pr[\mathcal{M}(y) \in \mathcal{S}] + \delta
\end{align} 

\remark In some cases, the $\delta$ is $0$. \\


\noindent\underline{\textbf{Problem3.}} \textsf{\color{blue} Describe the Dijkstra algorithm}.

\solution Algorithm \ref{algo:dijkstra} presents the Dijkstra algorithm in pseudo code format. As it will take a graph $G$ (weighted or unweighted) and query pair $(s, t)$ where $s$ represents the source and $t$ is terminal as input. And compute the shortest distance $d(s, t)$. In Line \ref{dijkstra:create}, we firstly create the minimum heap $\mathcal{Q}$ which stores each vertex $v$ and its distance to $s$: $d(v, s)$ as values ($\mathcal{Q}$ is maintained by $d(v, s)$), distance array $dis$ and Boole array $vis$ to record whether each vertex has been visited (\ie its shortest distance has been computed). Then \textsc{extractHeapTop} selects the vertex $u$ who has the shortest distance to $s$ at current stage (Line \ref{dijkstra:select}). Obviously, at the first run of the while loop, $u$ will be $s$. After extraction of $u$, it will be marked as visited in $vis$. Finally in Line \ref{dijkstra:update}, the algorithm will update $u$'s neighbors' distances and insert the neighbors in $\mathcal{Q}$ whose distances are updated.

\underline{\textsc{updateNeighbors.}} We go a little bit further in Line 4. Note $v$ as one of $u$'s neighbors. The procedure will firstly compare $dis[v]$ with $dis[u] + w(u, v)$. If the former is bigger, then the $dis[v]$ will be updated as $dis[u] + w(u, v)$ and $v$ with its updated distance will be appended to the heap $\mathcal{Q}$.

\remark At the beginning of \textsc{createAuxiliaryDataStructures}, all elements in $dis$ will be set to infinity and for entries in $vis$ will be $false$. But finally we will set $dis[s]$ to $0$ and then $s$ will be inserted into $\mathcal{Q}$. 
\begin{algorithm}
% \scriptsize
\caption{Dijkstra algorithm} \label{algo:dijkstra}
\begin{algorithmic}[1]
    \Require Data graph $G$ and query vertex pair $(s, t)$.
    \Ensure The shortest distance $d(s, t)$ between source $s$ and terminal $t$.
    \State $\mathcal{Q}, dis, vis$ $\leftarrow$ \textsc{createAuxiliaryDataStructures}($G$, $s$) \Comment{$\mathcal{Q}$ is a min heap} \label{dijkstra:create}
    \While{$\mathcal{Q}$ is \textbf{not} empty}
        \State $u$ $\leftarrow$ \textsc{extractHeapTop}($\mathcal{Q}$) \label{dijkstra:select}
        \State \textsc{updateNeighbors}($u$, $G$, $dis$, $vis$, $\mathcal{Q}$) \Comment{update the distance of $u$'s neighbors} \label{dijkstra:update}
    \EndWhile
    \State $d(s, t)$ $\leftarrow$ $dis[t]$
    \State \textbf{return} $d(s, t)$
\end{algorithmic}
\end{algorithm} \\


\noindent\underline{\textbf{Problem4.}} \textsf{\color{blue} Any interesting problems}.

\solution 

\begin{theorem} \label{thm:thm1}
    This is a theorem.
\end{theorem}

\begin{lemma} \label{lma:lma1}
    This is a lemma.
\end{lemma}

\begin{definition}{\textnormal{\textbf{\textsc{A Definition}}}} \label{def:def1}
    This is a definition.
\end{definition}

\begin{example} \label{exp:exp1}
    This is an example.
\end{example}

\begin{proof} \label{prf:prf1}
    This is a proof.
\end{proof}

\remark 


\end{document}
